%*******************************************************
% Abstract
%*******************************************************
%\renewcommand{\abstractname}{Abstract}
\pdfbookmark[1]{Abstract}{Abstract}
\begingroup
\let\clearpage\relax
\let\cleardoublepage\relax
\let\cleardoublepage\relax

\chapter*{Abstract}

Deep groundwater systems offer a rare window into the Earth's crust, revealing processes otherwise inaccessible by direct observation.
In tectonically active and granitic regions---particularly at plate boundaries---these waters often originate from great depths, heated by magmatic intrusions or radiogenic decay.
Transported to the surface through fault networks, they carry both geochemical signatures and microbial communities shaped by extreme subsurface conditions.
Together, these natural tracers provide powerful tools for investigating the interactions between hydrogeology, deep microbiology, and tectonics.

Recent advances in portable field technologies now enable high-frequency, long-term monitoring of dissolved gases—including noble gases (He, Ar, Kr) and reactive species (\ce{N2}, \ce{O2}, \ce{CH4}, \ce{CO2}, \ce{H2}).
However, challenges persist in geothermal systems where elevated temperatures cause water vapor condensation and clogging in gas analysis instruments.
Two new experimental approaches were developed to overcome these issues and were successfully implemented in contrasting geothermal environments---Lavey-les-Bains (Switzerland) and Beppu (Japan).
These methods allowed continuous monitoring of dissolved gas dynamics and revealed variability that could only be resolved with high temporal resolution.

In the Lavey-les-Bains system, year-round monitoring of water isotopes, noble gases, and conductivity identified seasonal mixing between deep thermal waters and shallower alluvial groundwater.
Microbial community profiles, however, remained remarkably stable across seasons, with clear depth-dependent structuring.
At \SI{200}{\metre} depth, sulfur-disproportionating Bacteria and Micrarchaeota dominated, whereas deeper zones were inhabited by sulfate- and iron-reducers or hydrogen oxidizers.
These findings point to temperature---and not seasonal hydrology---as the primary driver of microbial community composition in deep aquifers.

Extended time-series of gas data also revealed short-term geochemical anomalies coinciding with periods of increased seismic activity.
Shifts in gas ratios (e.g., \ce{CO2}/Ar, \ce{CH4}/Ar) were interpreted as indicators of stress-induced changes in groundwater mixing dynamics, likely caused by elastic deformation of subsurface pore networks.
These observations suggest a coupling between tectonic stress and deep groundwater chemistry, offering a geochemical fingerprint of seismic influence that precedes direct ground motion.

Together, these results demonstrate the potential of integrated geochemical and microbiological monitoring to advance our understanding of deep fluid systems.
The new methodological framework opens the door for broader applications in geothermal resource assessment (e.g., identification of geothermal systems), deep crustal microbial ecology, and the detection of crustal stress through hydrological signals.
Moreover, the findings lay the groundwork for functional analyses of deep microbial communities and underscore the role of deep aquifers as sensitive recorders of tectonic change.

\endgroup


\cleardoublepage%


\begingroup
\let\clearpage\relax
\let\cleardoublepage\relax
\let\cleardoublepage\relax

\begin{otherlanguage}{french}
\pdfbookmark[1]{Résumé}{Résumé}
\chapter*{Résumé}

Les systèmes d’eaux souterraines profondes offrent une fenêtre rare sur la croûte terrestre, révélant des processus autrement inaccessibles par l’observation directe.
Dans les régions tectoniquement actives et riches en granites --- en particulier aux frontières des plaques --- ces eaux proviennent souvent de grandes profondeurs, réchauffées par des intrusions magmatiques ou la désintégration radiogénique.
Acheminées vers la surface par des réseaux de failles, elles transportent à la fois des signatures géochimiques et des communautés microbiennes façonnées par des conditions extrêmes du sous-sol.
Ensemble, ces traceurs naturels constituent de puissants outils pour étudier les interactions entre hydrogéologie, microbiologie de la biosphère profonde et tectonique.

Les progrès récents des technologies portables de terrain permettent désormais une surveillance à haute fréquence et à long terme des gaz dissous, incluant les gaz rares (He, Ar, Kr) et des espèces réactives (\ce{N2}, \ce{O2}, \ce{CH4}, \ce{CO2}, \ce{H2}).
Cependant, des défis subsistent dans les systèmes géothermiques, où les températures élevées entraînent la condensation de vapeur d’eau et l’obstruction des instruments d’analyse.
Deux nouvelles approches expérimentales ont été développées pour contourner ces limitations et ont été mises en oeuvre avec succès dans des environnements géothermiques contrastés --- Lavey-les-Bains (Suisse) et Beppu (Japon).
Ces méthodes ont permis une surveillance continue de la dynamique des gaz dissous, révélant une variabilité perceptible uniquement grâce à une haute résolution temporelle.

Dans le système de Lavey-les-Bains, une surveillance tout au long de l'année des isotopes de l’eau, des gaz nobles et de la conductivité a mis en évidence un mélange saisonnier entre eaux thermales profondes et eaux alluviales plus superficielles.
En revanche, les profils des communautés microbiennes sont restés remarquablement stables d’une saison à l’autre, avec une structuration marquée selon la profondeur.
À \SI{200}{\metre} de profondeur, les communautés sont dominées par des bactéries dismutatrices du soufre et des Micrarchaeota, tandis que les zones plus profondes abritent des réducteurs de sulfate, de fer et/ou des oxydants d’hydrogène.
Ces résultats indiquent que la température --- et non l’hydrologie saisonnière --- constitue le principal facteur structurant les communautés microbiennes dans les aquifères profonds.

Des séries temporelles prolongées de données sur les gaz ont également révélé des anomalies géochimiques à court terme coïncidant avec des périodes d’activité sismique accrue.
Les variations observées dans les rapports gazeux (par exemple, \ce{CO2}/Ar, \ce{CH4}/Ar) ont été interprétées comme des indicateurs de changements de dynamique de mélange des eaux souterraines induits par le stress tectonique, probablement liés à la déformation élastique des réseaux de pores.
Ces observations suggèrent un couplage entre le stress tectonique et la chimie des eaux souterraines profondes, offrant ainsi une empreinte géochimique des influences sismiques précédant les mouvements du sol.

L’ensemble de ces résultats démontre le potentiel de la surveillance géochimique et microbiologique intégrée pour faire progresser notre compréhension des systèmes fluides profonds.
Le nouveau cadre méthodologique ouvre la voie à des applications élargies dans l’exploration géothermique (par exemple, l’identification de réservoirs profonds), l’écologie microbienne crustale, ainsi que la détection du stress tectonique via des signaux hydrologiques.
En outre, ces résultats jettent les bases d’analyses fonctionnelles des communautés microbiennes profondes et soulignent le rôle des aquifères profonds en tant qu’enregistreurs sensibles des changements tectoniques.

\end{otherlanguage}

\endgroup

\vfill


\cleardoublepage%


\begingroup
\let\clearpage\relax
\let\cleardoublepage\relax
\let\cleardoublepage\relax

\begin{otherlanguage}{ngerman}
\pdfbookmark[1]{Zusammenfassung}{Zusammenfassung}
\chapter*{Zusammenfassung}

Tiefe Grundwassersysteme bieten einen seltenen Einblick in die Erdkruste und ermöglichen Einblicke in Prozesse, die sonst der direkten Beobachtung entzogen sind.
In tektonisch aktiven und granitreichen Regionen --- insbesondere an Plattengrenzen --- stammen diese Wässer häufig aus grossen Tiefen und werden durch magmatische Intrusionen oder radiogenen Zerfall erhitzt.
Beim Aufstieg durch Störungszonen transportieren sie sowohl geochemische Signaturen als auch mikrobiologische Gemeinschaften, die unter extremen Bedingungen des Untergrunds entstanden sind.
Diese natürlichen Tracer stellen gemeinsam leistungsstarke Werkzeuge zur Untersuchung der Wechselwirkungen zwischen Hydrogeologie, Tiefenmikrobiologie und Tektonik dar.

Dank jüngster Fortschritte bei tragbaren Feldtechnologien ist heute eine hochfrequente, langfristige Überwachung gelöster Gase möglich --- darunter Edelgase (He, Ar, Kr) und reaktive Spezies (\ce{N2}, \ce{O2}, \ce{CH4}, \ce{CO2}, \ce{H2}).
In geothermischen Systemen bestehen jedoch weiterhin technische Herausforderungen, da hohe Temperaturen zur Kondensation von Wasserdampf und zur Verstopfung von Gasanalysegeräten führen.
Zwei neue experimentelle Ansätze wurden entwickelt, um diese Hürden zu überwinden, und erfolgreich in kontrastierenden geothermischen Umgebungen --- Lavey-les-Bains (Schweiz) und Beppu (Japan) --- eingesetzt.
Sie ermöglichen eine kontinuierliche Überwachung der gelösten Gasdynamik und offenbaren Variabilitäten, die nur durch eine hohe zeitliche Auflösung erkennbar sind.

Im System von Lavey-les-Bains zeigten ganzjährige Messungen von Wasserisotopen, Edelgasen und Leitfähigkeit eine saisonale Durchmischung zwischen tiefem Thermalwasser und oberflächennahem alluvialem Grundwasser.
Die mikrobiellen Gemeinschaften blieben jedoch über die Jahreszeiten hinweg bemerkenswert stabil und zeigten eine klare Tiefenstrukturierung.
In \SI{200}{\metre} Tiefe dominierten schwefeldisproportionierende Bakterien und Micrarchaeota, während in tieferen Schichten vor allem Sulfat- und Eisenreduzierer sowie Wasserstoffoxidierer vorkamen.
Dies deutet darauf hin, dass die Temperatur --- und nicht saisonale hydrologische Einflüsse --- der massgebliche Steuerfaktor für die Zusammensetzung mikrobieller Gemeinschaften in tiefen Aquiferen ist.

Erweiterte Zeitreihen gelöster Gase zeigten zudem kurzfristige geochemische Anomalien, die mit Phasen erhöhter seismischer Aktivität zusammenfielen.
Änderungen in Gasverhältnissen (z.B. \ce{CO2}/Ar, \ce{CH4}/Ar) wurden als Hinweise auf spannungsinduzierte Veränderungen der Grundwassermischung interpretiert, wahrscheinlich verursacht durch elastische Deformation der unterirdischen Porenräume.
Diese Beobachtungen legen eine Kopplung zwischen tektonischer Spannung und Tiefengrundwasserchemie nahe --- als geochemischer Fingerabdruck seismischer Einflüsse, der zeitlich einer spürbaren Bodenbewegung vorausgehen kann.

Insgesamt zeigen diese Ergebnisse das Potenzial integrierter geochemischer und mikrobiologischer Überwachung auf, um unser Verständnis tiefer Fluidsysteme entscheidend zu erweitern.
Der neue methodische Rahmen eröffnet vielfältige Anwendungen --- etwa in der geothermischen Ressourcenbewertung (z.B. Identifikation geothermischer Systeme), der Ökologie tiefenmikrobieller Gemeinschaften sowie der Detektion tektonischer Spannungen anhand hydrologischer Signale.
Darüber hinaus legen die Erkenntnisse den Grundstein für funktionelle Analysen mikrobieller Prozesse in der Tiefe und unterstreichen die Rolle tiefer Aquifere als empfindliche Archive tektonischer Veränderungen.

\end{otherlanguage}

\endgroup

\vfill